\subsubsection{Hypothesis 1: Education and Employment Status}

\paragraph{Data Preparation}
The raw form of the \textbf{Education level} variable includes 31 categories, one of which represents a non-applicable value. To make the data more workable, we first removed the NaN values, along with any categories labeled as non-applicable or unspecified. After cleaning, the education categories were grouped into six meaningful levels:
\begin{itemize}
    \item No Formal Education 
    \item Basic Education 
    \item Intermediate Education 
    \item Secondary Education 
    \item Vocational Education 
    \item Tertiary Education 
\end{itemize}

\paragraph{$\chi^2$ Test}
Once categorized, employment status and education level were loaded into a contingency table, and the $\chi^2$ test was applied. The $\chi^2$ test assesses the association between two categorical variables—in this case, education level and employment status. The null hypothesis ($H_0$) assumes no significant relationship between the two variables, while the alternative hypothesis ($H_1$) suggests a significant relationship.
The $\chi^2$ statistic is determined, which compares the observed frequencies to the expected frequencies under $H_0$.
The test statistic and accompanying $p$-value is compared to the critical value from the $\chi^2$ distribution table at $\alpha = 0.05$.

If the $p$-value is less than 0.05, the null hypothesis is rejected, indicating a significant relationship between education level and employment status.

\paragraph{Logistic Regression Analysis}
To investigate the impact of education level on employment likelihood, we employed the Logit model from the \texttt{statsmodels} library. Logistic regression is particularly suited for modeling binary outcomes, which is applicable in this context where employment status is categorized as either employed or not employed following previous preparation.

The logistic regression analysis proceeded through the following steps:
\begin{enumerate}
    \item Preprocessing the education categories with a Label Encoder and defining it as the primary predictor of employment status.
    \item The logistic regression model was fitted using the Logit function from \texttt{statsmodels}.
    \item Coefficients were estimated via maximum likelihood estimation (MLE), aiming to maximize the likelihood of observing the given data under the model.
    \item The estimated coefficients revealed the relationship between education level and employment status. A positive coefficient for an education level suggests increased odds of employment, whereas a negative coefficient indicates decreased odds.
    \item The significance of each coefficient was assessed using Wald tests and their associated \( p \)-values. A \( p \)-value less than \( \alpha = 0.05 \) denoted a significant influence of the corresponding education level on employment likelihood.
\end{enumerate}

This combined analysis through the $\chi^2$ test and logistic regression with the Logit model from \texttt{statsmodels} allowed for a comprehensive examination of education's impact on employment status, thus providing robust statistical evidence to support our hypothesis testing.
