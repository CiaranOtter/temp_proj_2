\section{Methodology and Implementation}

\subsection{Testing Methods}

A variety of statistical tests were employed to evaluate the impact of demographic factors on employment and salary outcomes. The methods applied are outlined below:

\paragraph{$\chi^2$ Test for Independence}
The $\chi^2$ test evaluates the association between two categorical variables. This test was applied to assess relationships between variables such as \textbf{education level}, \textbf{employment status}, \textbf{gender}, and \textbf{ethnicity}. It determines whether there is a significant dependence between these demographic factors and employment outcomes.

\paragraph{T-Test, ANOVA, and Welch’s Test}
For continuous variables like \textbf{salary} and \textbf{age}, t-tests and one-way ANOVA were used to compare means across categorical groups. ANOVA was utilized to identify statistically significant differences in means across three or more independent groups, particularly in analyzing relationships between factors like \textbf{education level} and \textbf{ethnicity}.

\paragraph{Q-Q Plot}
A Q-Q (Quantile-Quantile) plot is a graphical tool used to assess whether a dataset follows a particular theoretical distribution, typically the normal distribution. By plotting the quantiles of the sample data against the quantiles of the normal distribution, the Q-Q plot visually evaluates how well the data aligns with the theoretical distribution. Deviations from the straight line in the Q-Q plot indicate departures from normality, which can affect the validity of parametric tests.

\paragraph{Levene's Test}
Levene’s test is used to assess the equality of variances across different groups. It evaluates the null hypothesis that the variances are equal between the groups. This test is especially useful in situations where the assumption of homogeneity of variances is critical, such as in ANOVA. If Levene’s test shows significant differences in variances, alternative statistical tests like Welch’s ANOVA, which do not assume equal variances, may be more appropriate. When comparing the means of two groups, Welch’s t-test was used. Unlike the traditional t-test, Welch’s test does not assume equal variances, making it more robust in cases where group variances differ. This approach ensures greater accuracy in testing for significant differences between group means, particularly when the assumption of homogeneity of variance is violated.

\paragraph{Pearson’s Correlation}
Pearson's correlation coefficient (denoted as \textit{r}) measures the linear relationship between two continuous variables. It ranges from -1 to 1, where values close to 1 indicate a strong positive linear relationship, values close to -1 indicate a strong negative linear relationship, and values near 0 suggest no linear correlation. Pearson's correlation is commonly used to quantify the strength and direction of relationships between variables, such as age and salary.

\paragraph{Tukey’s Honest Significant Difference (HSD) Test}
Tukey’s HSD test is a post-hoc analysis used after an ANOVA to determine which specific groups' means are significantly different from each other. It controls for Type I error when making multiple comparisons, ensuring that the overall significance level is maintained. This test is useful when comparing multiple group means, such as examining whether different education categories have significantly different average salaries.

\subsection{Hypothesis Testing}

\subsubsection{Hypothesis 4: Gender and Salary/Education Level}

\paragraph{Data Preparation}
The dataset contained unspecified values in the \textbf{Gender} field, which were removed. The data was subsequently divided into two groups based on gender, focusing on salary analysis.

\paragraph{Q-Q Test}
A Q-Q test was conducted to assess the normality of salary distributions within each gender group, considering potential non-homogeneity of variance.

\paragraph{Mann-Whitney U Test}
Following the Q-Q test, the Mann-Whitney U test was applied to compare salary distributions between genders. This non-parametric test evaluates whether the ranks of salary values differ significantly between the

\subsubsection{Hypothesis 4: Gender and Salary/Education Level}

\paragraph{Data Preparation}
The dataset contained unspecified values in the \textbf{Gender} field, which were removed. The data was subsequently divided into two groups based on gender, focusing on salary analysis.

\paragraph{Q-Q Test}
A Q-Q test was conducted to assess the normality of salary distributions within each gender group, considering potential non-homogeneity of variance.

\paragraph{Mann-Whitney U Test}
Following the Q-Q test, the Mann-Whitney U test was applied to compare salary distributions between genders. This non-parametric test evaluates whether the ranks of salary values differ significantly between the

\subsubsection{Hypothesis 4: Gender and Salary/Education Level}

\paragraph{Data Preparation}
The dataset contained unspecified values in the \textbf{Gender} field, which were removed. The data was subsequently divided into two groups based on gender, focusing on salary analysis.

\paragraph{Q-Q Test}
A Q-Q test was conducted to assess the normality of salary distributions within each gender group, considering potential non-homogeneity of variance.

\paragraph{Mann-Whitney U Test}
Following the Q-Q test, the Mann-Whitney U test was applied to compare salary distributions between genders. This non-parametric test evaluates whether the ranks of salary values differ significantly between the

\subsubsection{Hypothesis 4: Gender and Salary/Education Level}

\paragraph{Data Preparation}
The dataset contained unspecified values in the \textbf{Gender} field, which were removed. The data was subsequently divided into two groups based on gender, focusing on salary analysis.

\paragraph{Q-Q Test}
A Q-Q test was conducted to assess the normality of salary distributions within each gender group, considering potential non-homogeneity of variance.

\paragraph{Mann-Whitney U Test}
Following the Q-Q test, the Mann-Whitney U test was applied to compare salary distributions between genders. This non-parametric test evaluates whether the ranks of salary values differ significantly between the

\subsubsection{Hypothesis 4: Gender and Salary/Education Level}

\paragraph{Data Preparation}
The dataset contained unspecified values in the \textbf{Gender} field, which were removed. The data was subsequently divided into two groups based on gender, focusing on salary analysis.

\paragraph{Q-Q Test}
A Q-Q test was conducted to assess the normality of salary distributions within each gender group, considering potential non-homogeneity of variance.

\paragraph{Mann-Whitney U Test}
Following the Q-Q test, the Mann-Whitney U test was applied to compare salary distributions between genders. This non-parametric test evaluates whether the ranks of salary values differ significantly between the


