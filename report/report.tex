\documentclass[conference]{IEEEtran}
\IEEEoverridecommandlockouts
% The preceding line is only needed to identify funding in the first footnote. If that is unneeded, please comment it out.
\usepackage{float}
\usepackage{cite}
\usepackage{booktabs}
\usepackage{amsmath,amssymb,amsfonts}
\usepackage{algorithmic}
\usepackage{graphicx}
\usepackage{textcomp}
\usepackage{xcolor}
\usepackage{placeins}
\usepackage{enumitem}
\usepackage{array}
\usepackage[utf8]{inputenc}

\def\BibTeX{{\rm B\kern-.05em{\sc i\kern-.025em b}\kern-.08em
    T\kern-.1667em\lower.7ex\hbox{E}\kern-.125emX}}
\begin{document}

\title{Template title\\
{\footnotesize \textsuperscript{*} 
  Statistical Analysis of Real-World Data: A Hypothesis Testing and Clustering Approach}
}


\author{\IEEEauthorblockN{Ciaran Otter}
\IEEEauthorblockA{\textit{Department of Computer science} \\
\textit{Principles of Data Science - CS771}\\
Stellenbosch University,
Stellenbosch, South Africa\\}
}

\maketitle

\begin{abstract}
\end{abstract}

\begin{IEEEkeywords}
\end{IEEEkeywords}

\section{Introduction}
The Adult Income dataset, sourced from the UCI Machine Learning Repository, contains demographic and income-related data for individuals in the United States~\cite{adult_2}. 
The dataset is often used to predict whether a person earns more than \$50,000 annually, based on features such as age, education, occupation, hours worked per week, and others. 
In this project, we aim to apply statistical techniques to analyze the relationship between various demographic factors and income. Specifically, we investigate the following hypotheses:


\begin{enumerate}
  \item Hypothesis 1: Does education level significantly affect the likelihood of earning more than \$50000 per year?
  \item Hypothesis 2: Is there a significant difference in hours worked per week between men and women?
  \item Hypothesis 3: Does race significantly affect the likelihood of earning more than \$50000 per year?
\end{enumerate}

Each hypothesis is tested using appropriate statistical models. 
Additionally, unsupervised learning techniques, including K-Means, DBSCAN, and Hierarchical Clustering, are applied to cluster individuals based on various features and explore patterns in the dataset.

\section{Methodology}
\subsection{Hypothesis Testing}
\subsubsection{Hypothesis 1: Does Education Level Affect Income?}

\begin{itemize}
\item Null Hypothesis ($H_0$): Education level does not significantly affect the likelihood of earning more than \$50000 per year.
\item Alternative Hypothesis ($H_1$): Education level significantly affects the likelihood of earning more than \$50000 per year.
\end{itemize}

To test this hypothesis, we performed a chi-square test of independence. 
Since both education level and income are categorical variables, the chi-square test is appropriate to evaluate whether there is a statistically significant association between these variables. 
Additionally, a logistic regression model was used to assess the predictive power of education level on income.


\subsubsection{Hypothesis 2: Does Gender Affect Hours Worked per Week?}

\begin{itemize}
\item Null Hypothesis ($H_0$): There is no significant difference in the number of hours worked per week between men and women.
\item Alternative Hypothesis ($H_1$): There is a significant difference in the number of hours worked per week between men and women.
\end{itemize}



An independent t-test was employed to compare the mean hours worked by men and women. 
This test is suitable because it compares the means of two independent groups, assuming the hours worked are normally distributed within each group.
\section{Results}

\include{data/hyp1_cont_table}
\begin{tabular}{lll}
\toprule
 & $\chi^2$ & $p$ \\
\midrule
Value & 905.29 & 0.00 \\
\bottomrule
\end{tabular}


\section{Conclusion}

\bibliographystyle{IEEEtran}
\bibliography{refs}

\end{document}
