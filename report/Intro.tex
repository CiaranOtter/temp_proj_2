\section{Introduction and Project Goal}

This project investigates the relationship between demographic factors such as highest education level, age, gender, and ethnicity with employment status and salary in South Africa. 
Using real-world 2023 household survey data found through stats SA's interactive data portal: `Nesstar'~\cite{StatsSA_GHS2023}, his investigation aims to conduct statistical hypothesis testing to ascertain whether these factors significantly influence employment outcomes and salary levels in the contemporary South African context. 
Additionally, clustering techniques will be employed to uncover underlying patterns within the dataset.

The dataset comprises 112 variables, providing a comprehensive overview of household statistics within the South African population. However, for this analysis, we will focus on several key variables that are particularly relevant to our research objectives:
\begin{itemize}
    \item \textbf{Education level}
    \item \textbf{Employment status}
    \item \textbf{Salary}
    \item \textbf{Age}
    \item \textbf{Gender}
    \item \textbf{Ethnicity}
\end{itemize}

\subsection{Hypotheses}
To guide our investigation, we have formulated the following hypotheses regarding the influence of demographic factors on employment status and salary:

\subsection{Hypothesis 1: Education and Employment Status}
\textbf{Research Question:} Does an individual's highest level of education influence their employment status?

\begin{itemize}
    \item \textbf{Null Hypothesis ($H_0$):} There is no significant relationship between education level and the likelihood of employment.
    \item \textbf{Alternative Hypothesis ($H_1$):} There is a statistically significant relationship between education level and the likelihood of employment.
\end{itemize}

\subsection{Hypothesis 2: Education and Salary}
\textbf{Research Question:} Does an individual's level of education have a significant impact on their salary?

\begin{itemize}
    \item \textbf{Null Hypothesis ($H_0$):} Education level does not significantly influence salary.
    \item \textbf{Alternative Hypothesis ($H_1$):} Education level has a statistically significant effect on salary.
\end{itemize}

\subsection{Hypothesis 3: Age and Salary}
\textbf{Research Question:} Does age affect the salary of an employed individual?

\begin{itemize}
    \item \textbf{Null Hypothesis ($H_0$):} Age has no significant effect on salary compensation.
    \item \textbf{Alternative Hypothesis ($H_1$):} Age significantly influences salary compensation.
\end{itemize}

\subsection{Hypothesis 4: Gender and Salary/Education Level}
\textbf{Research Question:} Does gender significantly affect salary and education level?

\textbf{Salary:}
\begin{itemize}
    \item \textbf{Null Hypothesis ($H_0$):} Gender has no significant impact on salary.
    \item \textbf{Alternative Hypothesis ($H_1$):} Gender has a statistically significant impact on salary.
\end{itemize}

\textbf{Education Level:}
\begin{itemize}
    \item \textbf{Null Hypothesis ($H_0$):} Gender does not significantly affect education level.
    \item \textbf{Alternative Hypothesis ($H_1$):} Gender significantly affects education level.
\end{itemize}

\subsection{Hypothesis 5: Ethnicity and Salary}
\textbf{Research Question:} Does ethnicity influence salary, controlling for education level?

\begin{itemize}
    \item \textbf{Null Hypothesis ($H_0$):} There are no significant differences in salary based on ethnicity when controlling for education level.
    \item \textbf{Alternative Hypothesis ($H_1$):} There are statistically significant differences in salary based on ethnicity, even when controlling for education level.
\end{itemize}
