\subsubsection{Hypothesis 3: Age and Salary}

\paragraph{Data Preparation}
The dataset was prepared in a manner consistent with previous hypothesis tests, ensuring data integrity and suitability for analysis.

\paragraph{Pearson Correlation}
Pearson's correlation coefficient was calculated to initially assess the strength and direction of the linear relationship between \textbf{Age} and \textbf{Salary}.

\paragraph{OLS Regression Test}
The data underwent preprocessing using a StandardScaler to standardize the \textbf{Age} variable. Subsequently, \textbf{Age} was designated as the primary predictor in the Ordinary Least Squares (OLS) regression model aimed at predicting individual salary.

To address potential heteroscedasticity, robust standard errors were implemented. A robust covariance model was employed to obtain robust covariance estimates, specifically using the HC1 covariance type, which adjusts for small sample sizes.

The coefficients of the model were estimated via the OLS method. The statistical significance of the coefficients was evaluated through \( p \)-values, with a threshold of \( p < 0.05 \) indicating significant relationships.

This methodology facilitated a comprehensive analysis of the relationship between age and salary, yielding insights critical for understanding socio-economic dynamics within the labor market.
